\documentclass[a4paper,11pt,dvipdfmx]{jsarticle}

\setcounter{secnumdepth}{4}

\usepackage{amsmath,amssymb,amsthm,mathrsfs,amsfonts,latexsym}
\usepackage{bm}
\usepackage{bbm}
\usepackage{tikz}
\usepackage{tikz-3dplot}
\usepackage{pgfplots}
\pgfplotsset{compat=1.18}
\usepackage[straightvoltages]{circuitikz}
\usetikzlibrary{intersections,calc,arrows.meta,quotes, angles}
\usepackage{fancybox}
\usepackage{enumerate}
\usepackage[hang,small,bf]{caption}
\usepackage[subrefformat=parens]{subcaption}
\captionsetup{compatibility=false}
\usepackage{ascmac}
\usepackage{multicol}
\usepackage{multirow}
\usepackage{nicematrix}
\NiceMatrixOptions{cell-space-limits = 1pt}
\usepackage{mathtools}
\usepackage{braket}
\usepackage[separate-uncertainty]{siunitx}
\usepackage{float}
\usepackage{makeidx}
\usepackage{wrapfig}
\usepackage{url}
\usepackage[numbers]{natbib}
\usepackage{booktabs}
\usepackage{tensor}
\usepackage{empheq}
\usepackage{ulem}
\usepackage{tcolorbox}
\tcbuselibrary{breakable, skins, theorems}
%\usepackage{physics}
\usepackage[dvipdfmx,% 欧文ではコメントアウトする
setpagesize=false,%
bookmarks=true,%
bookmarksdepth=tocdepth,%
bookmarksnumbered=true,%
colorlinks=true,%
linkcolor=blue,%
citecolor=red,%
urlcolor=magenta,%
pdftitle={},%
pdfsubject={},%
pdfauthor={},%
pdfkeywords={}%
]{hyperref}
\usepackage{pxjahyper}
\usepackage{cleveref}
\usepackage{autonum}

\DeclareMathOperator{\gradop}{grad}
\DeclareMathOperator{\divop}{div}
\DeclareMathOperator{\rotop}{rot}
\DeclareMathOperator{\curlop}{curl}
\let\Re\relax
\DeclareMathOperator{\re}{Re}
\let\Im\relax
\DeclareMathOperator{\im}{Im}

\crefname{equation}{式}{式}% {環境名}{単数形}{複数形} \crefで引くときの表示
\crefname{figure}{図}{図}
\crefname{table}{表}{表}
\crefname{algorithm}{Algorithm}{Algorithm}

\crefname{section}{第}{第}
\creflabelformat{section}{#2#1#3節}
\crefname{subsection}{第}{第}
\creflabelformat{subsection}{#2#1#3小節}
\crefname{footnote}{脚注}{脚注}
\creflabelformat{footnote}{#2#1#3}

\definecolor{burgundy}{rgb}{0.5, 0.0, 0.13}
\newtcbtheorem[number within=section]{mythm}{}%
{fonttitle=\gtfamily\sffamily\bfseries\upshape,colframe=burgundy,colback=burgundy!2!white,rightrule=0pt,leftrule=0pt,bottomrule=2pt,colbacktitle=burgundy,theorem style=standard,breakable,arc=0pt}{thm}
\crefname{thm}{定理}{定理}

\theoremstyle{definition}% 日本語用.定理とか斜体にならないようにする
\newtheorem{theorem}{定理}
\crefname{theorem}{定理}{定理}
\newtheorem*{theorem*}{定理}
\crefname{theorem*}{定理}{定理}
\newtheorem{lemma}{補題}
\crefname{lemma}{補題}{補題}
\newtheorem{corollary}{系}
\crefname{corollary}{系}{系}
\newtheorem*{proof*}{証明}
\crefname{proof*}{証明}{証明}
\newtheorem{assumption}{仮定}
\crefname{assumption}{仮定}{仮定}
\newtheorem{definition}[theorem]{定義}
\crefname{definition}{定義}{定義}
\newtheorem*{definition*}{定義}
\crefname{definition*}{定義}{定義}
\newtheorem{remark}{注意}
\crefname{remark}{注意}{注意}
\newtheorem{proposition}{命題}
\crefname{proposition}{命題}{命題}

\renewcommand{\thefootnote}{*\arabic{footnote}}
\newcommand{\crefpairconjunction}{と}
\newcommand{\crefrangeconjunction}{から}
\newcommand{\crefmiddleconjunction}{,}
\newcommand{\creflastconjunction}{,および}
\newcommand{\two}{\mathrm{I}\hspace{-1.2pt}\mathrm{I}}
\newcommand{\three}{\mathrm{I}\hspace{-1.2pt}\mathrm{I}\hspace{-1.2pt}\mathrm{I}}
\newcommand{\four}{\mathrm{I}\hspace{-1.5pt}\mathrm{V}}
\newcommand{\six}{\mathrm{V}\hspace{-1.5pt}\mathrm{I}}
\newcommand{\seven}{\mathrm{V}\hspace{-1.5pt}\mathrm{I}\hspace{-1.5pt}\mathrm{I}}
\newcommand{\eight}{\mathrm{V}\hspace{-1.5pt}\mathrm{I}\hspace{-1.5pt}\mathrm{I}\hspace{-1.5pt}\mathrm{I}}
\newcommand{\nine}{\mathrm{I}\hspace{-1.5pt}\mathrm{X}}
\newcommand{\ctext}[1]{\textcircled{\raise-0.2ex\hbox{#1}}}
\newcommand{\ketbra}[2]{\Ket{#1}\!\Bra{#2}}

\numberwithin{equation}{section}%式番号に節番号もふる
\numberwithin{theorem}{section} %定理番号に節番号もふる
%\mathtoolsset{showonlyrefs=true}%参照時以外は番号が振られない,hyperrefとかぶってる

%テキストの表示領域の調節
\setlength{\textwidth}{\paperwidth}
\addtolength{\textwidth}{-40truemm}
\setlength{\textheight}{\paperheight}
\addtolength{\textheight}{-45truemm}

%余白の調節
\setlength{\topmargin}{-25.4truemm}
\setlength{\evensidemargin}{-5.4truemm}
\setlength{\oddsidemargin}{-5.4truemm}
\setlength{\headheight}{17pt}
\setlength{\headsep}{10mm}
\addtolength{\headsep}{-17pt}
\setlength{\footskip}{15mm}
%\setlength{\columnseprule}{0.4pt}

\title{固体物理学\(\three\)}
\author{05231527 小島悠杜}
\date{}

\begin{document}

\maketitle

%\tableofcontents

A. トポロジカル物性

B. 強相関電子系

\section{物性物理におけるトポロジー}
\subsection{ベリー位相,ベリー接続,ベリー曲率}
\begin{itemize}
  \item 幾何学的位相
  
  まず,エネルギー縮退のない系を考える。

  \begin{align}
    \mathcal{H}(t)&: \ \text{時間依存するハミルトニアン} \\
    \left\{\ket{n(t)}\right\}&: \ \mathcal{H}(t)\text{の固有状態で正規直交完全系}
  \end{align}
  とすると
  \begin{equation}
    \mathcal{H}(t)\ket{n(t)}=E_n\ket{n(t)}.\tag{1.1.1}
  \end{equation}
  \(\sum_{n}\ketbra{n(t)}{n(t)}=\mathbbm{1}\)より
  \begin{equation}
    \ket{\psi(t)}=\sum_n C_n(t)\ket{n(t)}\tag{1.1.2}
  \end{equation}
  となる。ただし,\(C_n\coloneq\braket{n(t)|\psi(t)}\)。時間依存するシュレディンガー方程式より
  \begin{align}
    i\hbar\frac{d}{dt}\ket{\psi(t)}=\mathcal{H}(t)\ket{\psi(t)} \Longrightarrow i\hbar\sum_n\left(\dot{C_n}\ket{n(t)}+C_n\dot{\ket{n(t)}}\right)=\sum_nE_nC_n\ket{n(t)}.
  \end{align}
  両辺に\(\bra{m(t)}(m\neq n)\)を作用させると
  \begin{equation}
    i\hbar\dot{C_m}+i\hbar\sum_n C_n\braket{m|\dot{n}}=E_mC_m. \tag{1.1.3}
  \end{equation}
  ここで
  \begin{equation}
    \begin{dcases}
      \frac{d}{dt}\braket{n|m}=\braket{\dot{n}|m}+\braket{n|\dot{m}}=0 \Longrightarrow \braket{\dot{n}|m}=-\braket{n|\dot{m}} \tag{1.1.4,\ 1.1.5}\\
      \frac{d}{dt}\braket{n|\mathcal{H}|m}=\braket{\dot{n}|m}E_m+\braket{n|\dot{\mathcal{H}}|m}+\braket{n|\dot{m}}E_n=0 
    \end{dcases}
  \end{equation}
  \begin{equation}
    \therefore \braket{n|\dot{m}}=-\frac{\braket{n|\dot{\mathcal{H}}|m}}{(E_n-E_m)} \tag{1.1.6}
  \end{equation}
  (1.1.6)を(1.1.3)に代入すると
  \begin{equation}
    i\hbar\dot{C}_m+i\hbar C_m\braket{m|\dot{m}}+\sum_{n\neq m}\frac{\braket{m|\dot{\mathcal{H}}|n}}{E_n-E_m}C_n=C_mE_m
  \end{equation}
  \begin{equation}
    \therefore \frac{d}{dt}C_m(t)=\frac{1}{i\hbar}E_m(t)C_m(t)-C_m(t)\braket{m|\dot{m}}-\sum_{n\neq m}\frac{\braket{m|\dot{\mathcal{H}}|n}}{E_n-E_m}C_n(t)
  \end{equation}
  \(\sum_{n\neq m}\frac{\braket{m|\dot{\mathcal{H}}|n}}{E_n-E_m}\ll1\)と仮定して無視する(断熱近似)と
  \begin{equation}
    \frac{d}{dt}C_m(t)=\frac{1}{i\hbar}E_m(t)C_m(t)-\braket{m|\dot{m}}C_m(t).
  \end{equation}
  微分方程式を解くと
  \begin{equation}
    C_m(t)=C_m(0)\exp\left(i\theta_m(t)\right)\cdot\exp\left(i\gamma_m(t)\right)
  \end{equation}
  \begin{equation}
    \begin{dcases}
      \theta_m(t)\coloneq-\frac{1}{\hbar}\int_{0}^{t}E_m(t')dt' \ \text{\textcolor{blue}{動的位相 dynamical phase}} \\
      \gamma_m(t)\coloneq i\int_{0}^{t}\braket{m(t')|\dot{m}(t')}dt' \ \text{\textcolor{blue}{幾何学的位相 geometric phase}}
    \end{dcases}
  \end{equation}
  本当に位相?
  \begin{align}
    \theta_m&\cdots\text{明らかに実数} \\
    \gamma_m&\cdots\frac{d}{dt}\braket{m|m}=\braket{\dot{m}|m}+\braket{m|\dot{m}}=0\Longrightarrow 2\re\left[\braket{m|\dot{m}}\right]=0\Longrightarrow\braket{m|\dot{m}}\text{は純虚数}\Longrightarrow\gamma_m\text{は実数}
  \end{align}

  \item パラメータ\(\bm{R}(t)\)を通じてのみ\(t\)依存する場合
  
  \begin{equation}
    \begin{alignedat}{2}
      \mathcal{H}(t)&\longrightarrow & \mathcal{H}\left(\bm{R}(t)\right) & \\
      (1.1.9)&\longrightarrow & \theta_m(t)&\coloneq-\frac{1}{\hbar}{\int_{\bm{R_0}}^{\bm{R}}}_{C}E(\bm{R}')d\bm{R}' \\
      &&\gamma_m(t)&\coloneq i{\int_{\bm{R_0}}^{\bm{R}}}_C\braket{m(\bm{R}')|\nabla_{\bm{R}'}m(\bm{R}')}d\bm{R}'
    \end{alignedat}
    \tag{1.1.10}
  \end{equation}
  ここで\textcolor{blue}{ベリー接続}を\(\bm{A}_m(\bm{R})\coloneq i\braket{m(\bm{R})|\nabla_{\bm{R}}|m(\bm{R})}(1.1.11)\)と定義すると
  \begin{equation}
    \gamma_m(\bm{R})={\int_{\bm{R_0}}^{\bm{R}}}_C\bm{A}_m(\bm{R}')d\bm{R}'\tag{1.1.12}
  \end{equation}


  \item ベリー位相
  
  局所ゲージ変換\(\ket{(\bm{R})}\to\ket{n'(\bm{R})}=e^{i\phi(\bm{R})}\ket{n(\bm{R})}\)を考える。
  \begin{align}
    \bm{A}_n(\bm{R})\to \bm{A}'_{n}(\bm{R})&=ie^{-i\phi}\bra{m}\nabla_{\bm{R}}(e^{i\phi}\ket{m}) \\
    &=\bm{A}_m(\bm{R})-\nabla_{\bm{R}}\phi(\bm{R})
  \end{align}
  これはゲージ不変ではない。また,
  \begin{align}
    \gamma_m\to\gamma'_m&=\int_C(\bm{A}_m(\bm{R}')-\nabla_{\bm{R}'}\phi(\bm{R}'))d\bm{R}' \\
    &=\gamma_m(\bm{R})-\left[\phi(\bm{R})-\phi(\bm{R}_0)\right]
  \end{align}
  \(\Longrightarrow\)closed loopを考えれば良い!
  \begin{align}
    \gamma_n(\bm{R})&\coloneq \oint_C\sum_\mu \Braket{n(\bm{R})|\frac{\partial n}{\partial R_\mu}}dR_\mu \\
    &=i\oint_C \braket{n(\bm{R})|\nabla_{\bm{R}}|n(\bm{R})}d\bm{R} \\
    &=\oint_C \bm{A}(\bm{R})d\bm{R}\tag{1.1.13}
  \end{align}

  \item ベリー曲率\(\Omega\) 
  
  ストークスの定理が使えるように定義
  \begin{equation}
    \bm{\Omega}(\bm{R})\coloneq \nabla_{\bm{R}}\times \bm{A}(\bm{R})\tag{1.1.14}
  \end{equation}
  \begin{equation}
    \therefore \gamma_n=\oint\bm{A}_nd\bm{R}=\int_S\bm{\Omega} d\bm{S}
  \end{equation}
\end{itemize}

\begin{mythm}{まとめ}{}
Berry phase
\begin{align}
  \gamma_n(\bm{R})&\coloneq i\oint_C \braket{n(\bm{R}')|\nabla_{\bm{R}'}|n(\bm{R}')}d\bm{R}' \\
  &=\oint_C\bm{A}d\bm{R}'=\int_S\bm{\Omega}d\bm{S}
\end{align}

Berry connection 
\begin{equation}
  \bm{A}_n(\bm{R})\coloneq i\braket{n(\bm{R})|\nabla_{\bm{R}}|n(\bm{R})}
\end{equation}

Berry curvature 
\begin{equation}
  \bm{\Omega}(\bm{R})\coloneq \nabla_{\bm{R}}\times \bm{A}(\bm{R})
\end{equation}
(1.1.15)
\end{mythm}

\subsection{具体例}
\begin{enumerate}[1.]
  \item アハラノフボーム効果
  
  \begin{itemize}
    \item \(A\to B \to D\)と\(A\to C \to D\)で経路差ゼロ
    \item 磁場はソレノイドコイルの中のみに存在,経路上はゼロ
  \end{itemize}

  \(\Longrightarrow\)古典的には干渉は生じない?

  \textcolor{blue}{\(\Longrightarrow\)量子力学ではベクトルポテンシャルが位相に現れ,干渉を起こす。}

  ソレノイドから十分離れた\(\bm{r}=\bm{R}_0\)(経路上の点)でベクトルポテンシャルはゼロとする。これは基準点の設定なのでいつでもできる。そのときのシュレディンガー方程式の解を\(\psi_0(\bm{r}-\bm{R}_0)\)とする。ベクトルポテンシャルが存在しないとみなしたときには
  \begin{equation}
    -\frac{\hbar^2}{2m}\nabla^2\psi_0(\bm{r}-\bm{R}_0)=E\psi_0(\bm{r}-\bm{R}_0)\tag{1.2.1}
  \end{equation}
  が成り立つ。実際にはベクトルポテンシャルが経路上に存在するが,(1.2.1)と同じ固有値,(位相のみ異なる)固有状態のはず。
  \begin{equation}
    \therefore \psi(\bm{r})=e^{i\theta(\bm{r})}\psi_0(\bm{r}-\bm{R}_0)\tag{1.2.2}
  \end{equation}
  \begin{equation}
    \frac{1}{2m}\left(-i\hbar\nabla +\frac{e}{c}\bm{A}(\bm{r})\right)^2\psi(\bm{r})=E\psi(\bm{r})\tag{1.2.3}
  \end{equation}
  \(\theta\)は(1.2.3)をみたすように決める。
  \begin{equation}
    \theta(\bm{r})=-\frac{e}{\hbar c}\int_{\bm{R}_0}^{\bm{r}}\bm{A}(\bm{x})d\bm{x}
  \end{equation}
  \begin{align}
    \therefore \Delta\theta&\coloneq \theta_{A\to B \to D}-\theta_{A\to C \to D} \\
    &=-\frac{e}{\hbar c}\oint\bm{A}d\bm{x}=-\frac{e}{\hbar c}\Phi=-2\pi \frac{\Phi}{\Phi_0}\neq0 \ \left(\Phi_0\coloneq \frac{2\pi \hbar c}{|e|}: \ \text{\textcolor{blue}{磁束量子}}\right)(1.2.4)
  \end{align}
  \begin{itemize}
    \item ベリー位相との対応
    
    \(\bm{R}(t)\)を実空間の位置ベクトルとすると,

    ベリー接続\(\to\)ベクトルポテンシャル,ベリー曲率\(\to\)磁場,ベリー位相\(\to\)アハラノフボーム位相

    ソレノイドコイルの外,位置\(\bm{R}\)に置かれた箱中の電子を考える。
  \begin{equation}
    \left\{\frac{1}{2m}\left(-i\hbar\nabla+\frac{e}{c}\bm{A}\right)^2+U(\bm{r}-\bm{R})\right\}\psi(\bm{r})=E\psi(\bm{r})
  \end{equation}
  \(\bm{A}\)が全領域でゼロのときの解を\(\psi_0(\bm{r}-\bm{R})\)とすると
  \begin{equation}
    \psi(\bm{r})=\exp\left(-i\frac{e}{\hbar c}\int_{\bm{R}}^{\bm{r}}\bm{A}(\bm{x})d\bm{x}\right)\psi_0(\bm{r}-\bm{R})
  \end{equation}
  ※\(\psi_0\)にはglobal phaseの任意性があるので,箱内で\(\psi_0\)が実数になるように選ぶ。(\textcolor{red}{?})

  ベリー接続
  \begin{align}
    \bm{A}_B&=i\int d^3r \psi(\bm{r})\nabla_{\bm{R}}\psi(\bm{r}) \\
    &=\frac{e}{\hbar c}\bm{A}\int\psi^\ast\psi d^3r +i\int \psi_0^\ast\nabla_{\bm{R}}\psi_0d^3r
  \end{align}
  第2項目は\(\psi_0\)が実数であることから
  \begin{equation}
    \int \psi\nabla\psi d^3r=\frac{1}{2}\nabla\int\psi^2d^3r
  \end{equation}
  を考えると0。
  \begin{equation}
    \therefore \bm{A}_B=\frac{e}{\hbar c}\bm{A}
  \end{equation}
  ベリー位相
  \begin{equation}
    \gamma=\oint \bm{A}_B d\bm{R}=-\frac{e}{\hbar c}\int\bm{A} d\bm{S}=-2\pi\frac{\Phi}{\Phi_0}
  \end{equation}
  \end{itemize}

  \item 飛びあり
  
  これを使ってベリー接続等を計算してみる。
  \begin{align}
    i\braket{\bm{R}_+|\frac{\partial}{\partial \theta}|\bm{R}_-}&=i\braket{\bm{R}_-|\frac{\partial}{\partial\theta}|\bm{R}_-}0 \\
    i\braket{\bm{R}_+|\frac{\partial}{\partial\phi}|\bm{R}_+}&=
    \begin{pNiceMatrix}
    e^{i\phi}\cos\frac{\theta}{2} & \sin\frac{\theta}{2}
    \end{pNiceMatrix}
    \begin{pNiceMatrix}
    e^{-i\phi}\cos\frac{\theta}{2} \\
    0
    \end{pNiceMatrix}
    =\cos^2\frac{\theta}{2} \\
    i\braket{\bm{R}_-|\frac{\partial}{\partial\phi}|\bm{R}_-}&=\sin^2\frac{\theta}{2}
  \end{align}
  球面座標系では
  \begin{equation}
    \nabla=\bm{e}_R\frac{\partial}{\partial R}+\bm{e}_\theta\frac{1}{R}\frac{\partial}{\partial\phi}+\bm{e}_\phi\frac{1}{R\sin\theta}\frac{\partial}{\partial\phi}
  \end{equation}
  より
  \begin{align}
    \bm{A}^+&=i\braket{\bm{R}_+|\nabla|\bm{R}_+}=\frac{\cos^2\frac{\theta}{2}}{R\sin \theta}\bm{e}_\phi=\frac{1+\cos\theta}{2R\sin\theta}\bm{e}_\phi \\
    \bm{A}^-&=\frac{\sin\frac{\theta}{2}}{R\sin\theta}\bm{e}_\phi=\frac{1-\cos\theta}{2R\sin\theta}\bm{e}_\phi\tag{1.2.9}
  \end{align}
  \begin{align}
    \nabla\times\bm{A}&=\frac{1}{R^2\sin\theta}\left[\frac{\partial}{\partial\theta}(R\sin\theta A_\phi)-\frac{\partial}{\partial\phi}(RA_\theta)\right]\bm{e}_R+\frac{1}{R\sin\theta}\left[\frac{\partial}{\partial\phi}A_R-\frac{\partial}{\partial R}(R\sin\theta A_\phi)\right]\bm{e}_\theta+\frac{1}{R}\left[\frac{\partial}{\partial R}(RA_\theta)-\frac{\partial}{\partial\theta}A_R\right]\bm{e}_\phi
  \end{align}
  より
  \begin{align}
    \bm{\Omega}^\pm&=\nabla\times\bm{A}^\pm=\frac{\mp 1}{2R^2}\bm{e}_R=\mp\frac{\bm{R}}{2R^3}\tag{1.2.10}
  \end{align}
  ベリー位相は
  \begin{align}
    \gamma^\pm&=\int_{\text{球面}}\bm{\Omega}^\pm d\bm{S}=\mp\int\frac{1}{2R^2}R^2\sin\theta d\theta d\phi \\
    &=\mp\int\sin\theta d\theta d\phi \\
    &=\mp2\pi \tag{1.2.11}
  \end{align}
  電磁気学を思い出すと
  \begin{equation}
    \nabla \cdot\bm{E}=4\pi\rho, \quad \nabla\cdot\bm{B}=4\pi\rho_{\text{m}}\to\int\bm{B}d\bm{S}=4\pi Q_{\text{m}}
  \end{equation}
  なので,モノポールチャージ\(Q_{\text{m}}=\mp1/2\)を意味する。

  \begin{equation}
    \bm{A}^\pm=\frac{1\pm\cos\theta}{R\sin\theta}\bm{e}_\phi\rightarrow \theta=0,\pi\text{で}A^-,A^+\text{がそれぞれ定義できない}
  \end{equation}
  固有ベクトルが北極,南極で一意に定まっていないことに起因する。
  \begin{equation}
    \theta=\pi \quad \ket{\bm{R}_+}=\begin{pNiceMatrix}
      0 \\
      1
    \end{pNiceMatrix}, \quad \ket{\bm{R}_-}=\begin{pNiceMatrix}
      -e^{i\phi} \\
      0
    \end{pNiceMatrix}
  \end{equation}
  \begin{equation}
    \theta=0 \quad \ket{\bm{R}_+}=\begin{pNiceMatrix}
      e^{-i\phi} \\
      0
    \end{pNiceMatrix}, \quad \ket{\bm{R}_-}=\begin{pNiceMatrix}
      0 \\
      1
    \end{pNiceMatrix}
  \end{equation}
  よってゲージ\(\psi\)を\(\psi=-\phi\)として選んだ
  \begin{equation}
    \ket{\bm{R}_+}^\two=\begin{pNiceMatrix}
      \cos\frac{\theta}{2} \\
      e^{i\phi}\sin\frac{\theta}{2}
    \end{pNiceMatrix},\quad 
    \ket{\bm{R}_-}^\two=\begin{pNiceMatrix}
      \sin\frac{\theta}{2} \\
      e^{i\phi}\cos\frac{\theta}{2}
    \end{pNiceMatrix}
  \end{equation}
  を赤道でつぎはぎして問題を解決する。
  
\end{enumerate}

\subsection{ベリー曲率の別表式}
  \begin{equation}
    \bm{\Omega}^n=\nabla\times\bm{A}^n
  \end{equation}
  を成分表示する。
  \begin{align}
    \Omega^n_k&=\epsilon_{ijk}\partial_iA^n_j \\
    &=i\epsilon_{ijk}\partial_i\left(\braket{n(\bm{R})|\partial_jn(\bm{R})}\right) \\
    &=i\epsilon_{ijk}\left\{\braket{\partial_in(\bm{R})|\partial_jn(\bm{R})}+\braket{n(\bm{R})|\partial_i\partial_jn(\bm{R})}\right\} \because\text{対称と反対称の縮約}\\
    &=i\epsilon_{ijk}\braket{\partial_in(\bm{R})|\partial_jn(\bm{R})} \tag{1.3.1}
  \end{align}
  ここで\(\mathbbm{1}=\sum_m\ketbra{m(\bm{R})}{m(\bm{R})}\)を挿入
  \begin{align}
    \Omega^n_k&=i\sum_m\epsilon_{ijk}\braket{\partial_in|m}\braket{m|\partial_jn} \\
    &=i\sum_{m\neq m}\epsilon_{ijk}\braket{\partial_in|m}\braket{m|\partial_jn}\tag{1.3.2}
  \end{align}
  ここで\(\ket{n}\)は\(\mathcal{H}\)の固有状態なので
  \begin{equation}
    \mathcal{H}(\bm{R})\ket{n(\bm{R})}=E_n(\bm{R})\ket{n(\bm{R})}
  \end{equation}
  両辺\(\partial_i\)
  \begin{equation}
    \frac{\partial\mathcal{H}}{\partial R_i}\ket{n}+\mathcal{H}\ket{\partial_in}=\frac{\partial E_n}{\partial R_i}\ket{n}+E_n\ket{\partial_in}
  \end{equation}
  両辺\(\bra{m},m\neq n\)
  \begin{equation}
    \bra{m}\frac{\partial\mathcal{H}}{\partial R_i}\ket{n}+E_m\braket{m|\partial_in}=E_n\braket{m|\partial_in}
  \end{equation}
  \begin{equation}
    \therefore \braket{m|\partial_in}=\frac{\braket{m|\frac{\partial\mathcal{H}}{\partial R_i}|n}}{E_n-E_m}
  \end{equation}
  (1.3.2)に代入して
  \begin{equation}
    \Omega^n_i(\bm{R})=i\sum_{m\neq n}\epsilon_{ijk}\frac{\braket{n|\frac{\partial\mathcal{H}}{\partial R_i}|m}\braket{m|\frac{\partial \mathcal{H}}{\partial R_j}|n}}{(E_n-E_m)^2}\tag{1.3.3}
  \end{equation}
  または,ベクトル表示で
  \begin{equation}
    \bm{\Omega}^n=i\sum_{m\neq n}\frac{\braket{n|\nabla\mathcal{H}|m}\times\braket{m|\nabla\mathcal{H}|n}}{(E_n-E_m)^2}\tag{1.3.4}
  \end{equation}

  \subsection{Bloch状態のベリー曲率}
  物質中の電子は波数\(\bm{k}\)で特徴づけられるBloch状態
  \begin{equation}
    \psi_{n,\bm{k}}(\bm{r})=e^{i\bm{k}\cdot\bm{r}}u_{n,\bm{k}}(\bm{r})\tag{1.4.1}
  \end{equation}
%\bibliographystyle{aip}
%\bibliography{}
\end{document}